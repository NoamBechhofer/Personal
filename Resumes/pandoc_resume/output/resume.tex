% Copyright 2013 Christophe-Marie Duquesne <chmd@chmd.fr>
% Copyright 2014 Mark Szepieniec <http://github.com/mszep>
% 
% ConText style for making a resume with pandoc. Inspired by moderncv.
% 
% This CSS document is delivered to you under the CC BY-SA 3.0 License.
% https://creativecommons.org/licenses/by-sa/3.0/deed.en_US

\startmode[*mkii]
  \enableregime[utf-8]  
  \setupcolors[state=start]
\stopmode

\setupcolor[hex]
\definecolor[titlegrey][h=757575]
\definecolor[sectioncolor][h=397249]
\definecolor[rulecolor][h=9cb770]

% Enable hyperlinks
\setupinteraction[state=start, color=sectioncolor]

\setuppapersize [A4][A4]
\setuplayout    [width=middle, height=middle,
                 backspace=20mm, cutspace=0mm,
                 topspace=10mm, bottomspace=20mm,
                 header=0mm, footer=0mm]

%\setuppagenumbering[location={footer,center}]

\setupbodyfont[11pt, helvetica]

\setupwhitespace[medium]

\setupblackrules[width=31mm, color=rulecolor]

\setuphead[chapter]      [style=\tfd]
\setuphead[section]      [style=\tfd\bf, color=titlegrey, align=middle]
\setuphead[subsection]   [style=\tfb\bf, color=sectioncolor, align=right,
                          before={\leavevmode\blackrule\hspace}]
\setuphead[subsubsection][style=\bf]

\setuphead[chapter, section, subsection, subsubsection][number=no]

%\setupdescriptions[width=10mm]

\definedescription
  [description]
  [headstyle=bold, style=normal,
   location=hanging, width=18mm, distance=14mm, margin=0cm]

\setupitemize[autointro, packed]    % prevent orphan list intro
\setupitemize[indentnext=no]

\defineitemgroup[enumerate]
\setupenumerate[each][fit][itemalign=left,distance=.5em,style={\feature[+][default:tnum]}]

\setupfloat[figure][default={here,nonumber}]
\setupfloat[table][default={here,nonumber}]

\setuptables[textwidth=max, HL=none]
\setupxtable[frame=off,option={stretch,width}]

\setupthinrules[width=15em] % width of horizontal rules

\setupdelimitedtext
  [blockquote]
  [before={\setupalign[middle]},
   indentnext=no,
  ]


\starttext

\section[title={Noam Bechhofer},reference={noam-bechhofer}]

\startplacetable[location=none]
\startxtable
\startxtablebody[body]
\startxrow
\startxcell[align=right] 535 W 112 St \stopxcell
\startxcell[align=left] noam.bechhofer@columbia.edu \stopxcell
\stopxrow
\startxrow
\startxcell[align=right] New York, NY 10025 \stopxcell
\startxcell[align=left] linkedin.com/in/noam-bechhofer/ \stopxcell
\stopxrow
\startxrow
\startxcell[align=right] USA \stopxcell
\startxcell[align=left] github.com/NoamBechhofer \stopxcell
\stopxrow
\stopxtablebody
\startxtablefoot[foot]
\startxrow
\startxcell[align=right] (845) 263 - 4510 \stopxcell
\startxcell[align=left]  \stopxcell
\stopxrow
\stopxtablefoot
\stopxtable
\stopplacetable

\subsection[title={Education},reference={education}]

\startdescription{2010-2014 (expected)}
  {\bf PhD, Computer Science}; Awesome University (MyTown)

  {\em Thesis title: Deep Learning Approaches to the Self-Awesomeness
  Estimation Problem}
\stopdescription

\startdescription{2007-2010}
  {\bf BSc, Computer Science and Electrical Engineering}; University of
  HomeTown (HomeTown)

  {\em Minor: Awesomeology}
\stopdescription

\subsubsection[title={Columbia University, School of General
Studies},reference={columbia-university-school-of-general-studies}]

New York, NY - Graduating May 2025\crlf
Bachelor of Arts - Computer Science (Concentration: Software Systems),
GPA: 3.77 Relevant Coursework: - Data Structures: Arrays, FIFO/FILO,
Lists, Queues, Trees, Hashtables, Graphs, and related algorithms -
Discrete Mathematics: Proof Techniques, Set Theory, Number Theory,
Combinatorics, Probability, Graphs - Fundamentals of Computer Systems:
Boolean Algebra, Circuit Design, State Machines, Assembly Programming,
CPU Design - Advanced Programming: Systems Programming, C, TCP/IP
Networking, HTTP Server, Concurrency, UNIX - Operating Systems:
Scheduling, Memory Management, File Systems, Extensive Team-based Linux
Kernel hacking - Design Using C++: Taught by Bjarne Stroustrup. System
Design, Best Practices, Modern C++

\subsubsection[title={University of
Auckland},reference={university-of-auckland}]

Auckland, New Zealand - Feb 2023 - June 2023 Certificate of Proficiency
- Study Abroad Relevant Coursework: - Object Oriented Software
Development: UML Modelling, SOLID, Common Object-Oriented Design
Patterns, Java GUI - Mathematical Foundations of Computer Science:
Finite Automata, Turing Machines, Computability, Complexity

\#\# Skills \#\#\# Programming C, Rust, C++, Java, JavaScript,
TypeScript, Python \#\#\# Software Git \#\#\# Soft Skills AGILE,
Pragmatic Programming, Systems Programming, Teamwork \#\#\# Language
English (native), Hebrew (conversational)

\#\# Experience \#\#\# Teaching Assistant Columbia University, New York,
NY, Sep 2022 - Present - Assist in teaching a 500+ student Introduction
to Computer Science course taught in Java - Hold office hours to assist
students with coursework, programming and problem solving - Monitor and
provide assistance on course discussion board - Grade programming and
written assignments

\subsubsection[title={Backend Developer},reference={backend-developer}]

WDCC - Web Development & Consulting Club, Auckland, New Zealand, Apr
2023 -- Aug 2023 - Contributed to back end for a student club event
registration system - Integrated Stripe Checkout API - Developed
internal HTTP API between frontend and backend

\#\#\# Intern SkylerAI, Tel Aviv, Israel (remote), Jun 2022 - Aug 2022 -
Developed a tool to cluster SMS messages by intent - Implemented model
for recognizing trends in conversation using NLP software

\#\# Activities \#\#\# External Manager Beit Ephraim Housing Co-op,
2022-23 - Participated in House executive board, voting on policies and
regulations for residents - Liaison to landlord (Columbia University
Residential office) - Facilitated move-ins, move-outs, and transfers -
Communicated with Columbia Facilities for repairs and upkeep

\stoptext
